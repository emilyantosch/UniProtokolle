\documentclass{article}


\usepackage{circuitikz} %Für die Schaltpläne
\usepackage[T1]{fontenc}
\usepackage[utf8]{inputenc}
\usepackage{amsmath}

\usepackage{fancyhdr}
\usepackage{lettrine}
\usepackage{hyperref}
\usepackage{subcaption}
\usepackage{tikz}
\usepackage{cite}
\usepackage{listings}
\usepackage[nottoc, numbib]{tocbibind}
\usepackage{../assets/scripts/tex/color-env}
\usepackage[ngerman]{babel}
\input{../assets/scripts/tex/structure.tex}


\usetikzlibrary{shapes}
    \usetikzlibrary{arrows}
    \usetikzlibrary{arrows.meta,topaths}
    \usetikzlibrary{bending}
    \usetikzlibrary{calc}
\title{Elektrotechnik 1 Praktikum 1}


\usepackage[
  includehead,
  headheight = 17mm,
  footskip = \dimexpr\headsep+\ht\strutbox\relax,
  tmargin = 0mm,
  bmargin = \dimexpr17mm+2\ht\strutbox\relax,
]{geometry}

\usepackage{anyfontsize}

\usepackage{xcolor}

\definecolor{DarkGreenBlue}{HTML}{264653}
\definecolor{LightGreenBlue}{HTML}{2A9D8F}
\definecolor{LightOrange}{HTML}{E9C46A}
\definecolor{DarkOrange}{HTML}{F4A261}
\definecolor{RedOrange}{HTML}{E76F51}
\definecolor{BrightRed}{HTML}{D62828}
\definecolor{DeepBlue}{HTML}{003049}

\definecolor{codegreen}{rgb}{0,0.6,0}
\definecolor{codegray}{rgb}{0.5,0.5,0.5}
\definecolor{codepurple}{rgb}{0.58,0,0.82}
\definecolor{backcolour}{rgb}{0.95,0.95,0.92}

\lstdefinestyle{code}{
    backgroundcolor=\color{backcolour},   
    commentstyle=\color{codegreen},
    keywordstyle=\color{magenta},
    numberstyle=\tiny\color{codegray},
    stringstyle=\color{codepurple},
    basicstyle=\ttfamily\footnotesize,
    breakatwhitespace=false,         
    breaklines=true,                 
    captionpos=b,                    
    keepspaces=true,                 
    numbers=left,                    
    numbersep=5pt,                  
    showspaces=false,                
    showstringspaces=false,
    showtabs=false,                  
    tabsize=2
}

\lstset{style=code,escapeinside=$ $}
\let\origthelstnumber\thelstnumber
\makeatletter
\newcommand*\Suppressnumber{%
  \lst@AddToHook{OnNewLine}{%
    \let\thelstnumber\relax%
     \advance\c@lstnumber-\@ne\relax%
    }%
}

\newcommand*\Reactivatenumber[1]{%
  \setcounter{lstnumber}{\numexpr#1-1\relax}
  \lst@AddToHook{OnNewLine}{%
   \let\thelstnumber\origthelstnumber%
   \refstepcounter{lstnumber}
  }%
}



\pagestyle{fancy}
\fancyhead[L]{\leftmark}
\fancyhead[R]{}
\fancyfoot[L]{}
\fancyfoot[C]{\thepage}
\fancyfoot[R]{\includegraphics[scale=0.2]{../assets/images/haw.jpg}}
\renewcommand\headrulewidth{0.5pt}


\begin{document}


\thispagestyle{empty}
\begin{tikzpicture}[overlay,remember picture]
  \thispagestyle{empty}
  \fill[black!2] (current page.south west) rectangle (current page.north east);

  \begin{scope}[transform canvas ={rotate around ={45:($(current page.north west)+(-.5,-6)$)}}]

    \shade[rounded corners=18pt, left color=DarkGreenBlue, right color=LightGreenBlue] ($(current page.north west)+(-.5,-6)$) rectangle ++(9,1.5);

  \end{scope}

  \begin{scope}[transform canvas ={rotate around ={45:($(current page.north west)+(.5,-10)$)}}]

    \shade[rounded corners=18pt, left color=LightOrange,right color=DarkOrange] ($(current page.north west)+(0.5,-10)$) rectangle ++(15,1.5);

  \end{scope}

  \begin{scope}[transform canvas ={rotate around ={45:($(current page.north west)+(0.5,-10)$)}}]

    \shade[rounded corners=8pt, right color=DarkOrange, left color=LightOrange] ($(current page.north west)+(1.5,-9.55)$) rectangle ++(7,.6);

  \end{scope}

  \begin{scope}[transform canvas ={rotate around ={45:($(current page.north)+(-1.5,-3)$)}}]

    \shade[rounded corners=12pt, left color=DeepBlue!80, right color=DeepBlue!60] ($(current page.north)+(-1.5,-3)$) rectangle ++(9,0.8);

  \end{scope}

  \begin{scope}[transform canvas ={rotate around ={45:($(current page.north)+(-3,-8)$)}}]

    \shade[rounded corners=28pt, left color=BrightRed, right color=BrightRed!80] ($(current page.north)+(-3,-8)$) rectangle ++(15,1.8);

  \end{scope}

  \begin{scope}[transform canvas ={rotate around ={45:($(current page.north west)+(4,-15.5)$)}}]

    \shade[rounded corners=25pt, left color=RedOrange, right color=DarkOrange] ($(current page.north west)+(4,-15.5)$) rectangle ++(30,1.8);

  \end{scope}

  \begin{scope}[transform canvas ={rotate around ={45:($(current page.north west)+(13,-10)$)}},]

    \shade[rounded corners=22pt, left color=DeepBlue,right color=DarkGreenBlue] ($(current page.north west)+(13,-10)$) rectangle ++(15,1.5);

  \end{scope}

  \begin{scope}[transform canvas ={rotate around ={45:($(current page.north west)+(18,-8)$)}},]

    \shade[rounded corners=8pt, left color=DarkOrange] ($(current page.north west)+(18,-8)$) rectangle ++(15,0.6);

  \end{scope}

  \begin{scope}[transform canvas ={rotate around ={45:($(current page.north west)+(19,-5.65)$)}},]

    \shade[rounded corners=12pt, left color=RedOrange] ($(current page.north west)+(19,-5.65)$) rectangle ++(15,0.8);

  \end{scope}

  \begin{scope}[transform canvas ={rotate around ={45:($(current page.north west)+(20,-9)$)}}]

    \shade[rounded corners=20pt, left color=BrightRed, right color=BrightRed!80] ($(current page.north west)+(20,-9)$) rectangle ++(14,1.2);

  \end{scope}

  \draw[ultra thick,gray] ($(current page.center)+(5,2)$) -- ++(0,-3cm) node[midway,left=0.25cm,text width=5cm,align=right,black!75]{{\fontsize{25}{30} \selectfont \bf MP-P2\\[10pt] Gruppe 4}} node[midway,right=0.25cm,text width=6cm,align=left,orange]{{\fontsize{70}{86} \selectfont 2021}};

  \node at ($(current page.center)+(0,-4)$) {{\fontsize{40}{72} \selectfont Timer und Interrupts}};

  \node[text width=8cm,align=center] at ($(current page.center)+(0,-6.5)$) {{\fontsize{16}{20} \selectfont \textcolor{orange}{ \bf \today}} \\[3pt] Florian Tietjen 2519584\\[3pt] Emily Antosch 2519935};

\end{tikzpicture}

\newpage


\tableofcontents

\listoffigures

\lstlistoflistings

\newpage

\section{Einführung}

Im dritten Praktikum wollen wir uns mit dem A/D-Wandler des TivaWare-Boards auseinandersetzen. Dabei wollen wir sowohl herausfinden, wie man mit externen Peripheriegeräten arbeiten kann, als auch das interne A/D-Modul effektiv nutzen. Darüberhinaus interessiert uns auch der PWM-Modus des Timers als mögliche Dimmung einer LED basierend auf dem analogen Signal eines Sensors (in unserem Beispiel verwenden wir einen Joystick).

\section{Aufgabe 1: Externe D/A-Wandler mit Treppenverfahren}

In unserer ersten Aufgabe benutzen wir das in der Vorlesung behandelte Treppenverfahren, um den Spannungswert eines Komparators zu ermitteln und diesen mittels BCD-Code auf einem Display auszugeben. Wir möchten zudem dann die Richtigkeit unseres Ergebnisses überprüfen, indem wir die verschiedenen Spannungen auf dem Oszilloskop anzeigen lassen. Die Triggerspannung erhalten wir dabei von Pin $PL(2)$.

Wir verwenden dabei folgenden Code:

\begin{lstlisting}[language=c, caption={stepfnc.c zur Approximation der Spannung $U_{E}$ mit dem Treppenverfahren}, captionpos=b]
#include<stdio.h>
#include<stdint.h>
#include "inc/tm4c1294ncpdt.h"

void init_port(void);
void init_timer(void);
void init_adc(void);
unsigned int read_adc();

int main(void){

    unsigned int adc_value = 0; // ADC value variable
    unsigned short int minmax = 0; // If 0 then min, if 1 then max
    init_port();
    init_timer();
    init_adc();

    while (1)
    {
        ADC0_PSSI_R = 0x0001; // enable ADC0 SS0
        adc_value = read_adc(); // Read ADC value
        printf("Spannung: %d\n",adc_value); // Print ADC value

        if(!(GPIO_PORTM_DATA_R &= (1<<1))){
            // If the button is not pressed
            TIMER2_TAMATCHR_R = 8000 + (2000 - adc_value) * 4 * (95.0 / 100.0); // Set the match value for PWM from 95% to 5%
        }
        else{
            // If the button is pressed
            minmax = !minmax;
            TIMER2_TAMATCHR_R = 8000 + (2 - (minmax ? 0x04 : 0x00)) * 4000 * (95.0 / 100.0); // Set the match value for PWM for either min or max
            while(GPIO_PORTM_DATA_R &= (1<<1)); // Wait until the button is released
        }
    }
}

// Initialize the port
void init_port(void){
    // Enable clock for port M and Port E
    SYSCTL_RCGCGPIO_R |= (1 << 4) | (1 << 11);
    // Ready?
    while(!(SYSCTL_PRGPIO_R & ((1<<4)|(1<<11))));
    // Enable clock for ADC0
    SYSCTL_RCGCADC_R |= 0x01;
    // Ready?
    while(!(SYSCTL_PRADC_R & 0x01));
    // Port E setup
    GPIO_PORTE_AHB_AFSEL_R |= 0x01;
    GPIO_PORTE_AHB_DEN_R &= ~0x01;
    GPIO_PORTE_AHB_AMSEL_R |= 0x01;
    // Port M setup
    GPIO_PORTM_DEN_R = 0x03;
    GPIO_PORTM_DIR_R = 0x01;
    GPIO_PORTM_DATA_R &= ~(1<<0);
    GPIO_PORTM_AFSEL_R = 0x01;
    GPIO_PORTM_PCTL_R = 0x03;
    GPIO_PORTM_PUR_R = 0x02;
}
// Initialize the timer
void init_timer(void){
    // Enable clock for timer 2
    SYSCTL_RCGCTIMER_R |= (1<<2);
    // Ready?
    while (!(SYSCTL_PRTIMER_R & (1<<2)));
    // Timer 2 setup
    TIMER2_CTL_R &= ~(1<<0);
    TIMER2_CFG_R = 0x04;
    TIMER2_TAMR_R = (1<<3) | 0x02;
    TIMER2_TAILR_R = 16000-1;
    TIMER2_TAMATCHR_R = 16000/2 - 1;
    GPIO_PORTM_DATA_R |= (1<<0);
    TIMER2_CTL_R |= (1<<0);
}
// Initialize the ADC
void init_adc(void){

    unsigned int waitcycle = 0;
    // Disable ADC0 SS0
    ADC0_ACTSS_R &= ~0x0F;
    // Magic code
    SYSCTL_PLLFREQ0_R |= (1<<23);
    while(!(SYSCTL_PLLSTAT_R & 0x01));
    ADC0_CC_R |= 0x01;
    waitcycle++;
    SYSCTL_PLLFREQ0_R &= ~(1<<23);
    // Set Sequencer 0 to sample channel 0, AIN3
    ADC0_SSMUX0_R |= 0x03;
    ADC0_SSCTL0_R |= 0x02;
    // Enable ADC0 SS0
    ADC0_ACTSS_R |= 0x01;
}
// Read the ADC
unsigned int read_adc(){
    unsigned int result = 0;
    while(ADC0_SSFSTAT0_R & (1<<8));
    result = (unsigned int)ADC0_SSFIFO0_R * 5000 / 4095;
    return result;
}

\end{lstlisting}


\section{Aufgabe 3: Dimmen einer LED mithilfe des internen A/D-Wandlers}

In der nächsten Aufgabe wollen wir uns mit dem internen A/D-Wandler befassen. Mit diesem messen wir die Ausgangsspannung eines Joysticks in der Y-Achse. Einen Pin des Ports M verbinden wir dann mit einer LED, um diese mit einem der Timer im PWM-Modus zu dimmen. Eine Bewegung führt daher zu einer Erhöhung oder Verringerung der Helligkeit der LED. Dafür verwenden wir den

\end{document}
