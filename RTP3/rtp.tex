\documentclass{report}


\usepackage[T1]{fontenc}
\usepackage[utf8]{inputenc}
\usepackage{amsmath}


\usepackage{enumerate}

\usepackage{graphicx}
\usepackage{fancyhdr}
\usepackage{lettrine}
\usepackage{hyperref}
\usepackage{subcaption}
\usepackage{tikz}
\usepackage{cite}
\usepackage{listings}
\usepackage[nottoc, numbib]{tocbibind}
\usepackage{../assets/scripts/tex/color-env}
\usepackage[ngerman]{babel}
\usepackage[Glenn]{fncychap}
\usepackage{trfsigns}
\input{../assets/scripts/tex/structure.tex}


\usetikzlibrary{shapes}
    \usetikzlibrary{arrows}
    \usetikzlibrary{arrows.meta,topaths}
    \usetikzlibrary{bending}
    \usetikzlibrary{calc}
\title{Elektrotechnik 1 Praktikum 1}


\usepackage[
  includehead,
  headheight = 17mm,
  footskip = \dimexpr\headsep+\ht\strutbox\relax,
  tmargin = 0mm,
  bmargin = \dimexpr17mm+2\ht\strutbox\relax,
]{geometry}

\usepackage{anyfontsize}

\usepackage{xcolor}

\definecolor{DarkGreenBlue}{HTML}{264653}
\definecolor{LightGreenBlue}{HTML}{2A9D8F}
\definecolor{LightOrange}{HTML}{E9C46A}
\definecolor{DarkOrange}{HTML}{F4A261}
\definecolor{RedOrange}{HTML}{E76F51}
\definecolor{BrightRed}{HTML}{D62828}
\definecolor{DeepBlue}{HTML}{003049}

\definecolor{codegreen}{rgb}{0,0.6,0}
\definecolor{codegray}{rgb}{0.5,0.5,0.5}
\definecolor{codepurple}{rgb}{0.58,0,0.82}
\definecolor{backcolour}{rgb}{0.95,0.95,0.92}

\lstdefinestyle{code}{
    backgroundcolor=\color{backcolour},
    commentstyle=\color{codegreen},
    keywordstyle=\color{magenta},
    numberstyle=\tiny\color{codegray},
    stringstyle=\color{codepurple},
    basicstyle=\ttfamily\footnotesize,
    breakatwhitespace=false,
    breaklines=true,
    captionpos=b,
    keepspaces=true,
    numbers=left,
    numbersep=5pt,
    showspaces=false,
    showstringspaces=false,
    showtabs=false,
    tabsize=2
}

\lstset{style=code}

\pagestyle{fancy}
\fancyhead[L]{\leftmark}
\fancyhead[R]{}
\fancyfoot[L]{}
\fancyfoot[C]{\thepage}
\fancyfoot[R]{\includegraphics[scale=0.2]{../assets/images/haw.jpg}}
\renewcommand\headrulewidth{0.5pt}


\begin{document}


\thispagestyle{empty}
\begin{tikzpicture}[overlay,remember picture]
  \thispagestyle{empty}
  \fill[black!2] (current page.south west) rectangle (current page.north east);

  \begin{scope}[transform canvas ={rotate around ={45:($(current page.north west)+(-.5,-6)$)}}]

    \shade[rounded corners=18pt, left color=DarkGreenBlue, right color=LightGreenBlue] ($(current page.north west)+(-.5,-6)$) rectangle ++(9,1.5);

  \end{scope}

  \begin{scope}[transform canvas ={rotate around ={45:($(current page.north west)+(.5,-10)$)}}]

    \shade[rounded corners=18pt, left color=LightOrange,right color=DarkOrange] ($(current page.north west)+(0.5,-10)$) rectangle ++(15,1.5);

  \end{scope}

  \begin{scope}[transform canvas ={rotate around ={45:($(current page.north west)+(0.5,-10)$)}}]

    \shade[rounded corners=8pt, right color=DarkOrange, left color=LightOrange] ($(current page.north west)+(1.5,-9.55)$) rectangle ++(7,.6);

  \end{scope}

  \begin{scope}[transform canvas ={rotate around ={45:($(current page.north)+(-1.5,-3)$)}}]

    \shade[rounded corners=12pt, left color=DeepBlue!80, right color=DeepBlue!60] ($(current page.north)+(-1.5,-3)$) rectangle ++(9,0.8);

  \end{scope}

  \begin{scope}[transform canvas ={rotate around ={45:($(current page.north)+(-3,-8)$)}}]

    \shade[rounded corners=28pt, left color=BrightRed, right color=BrightRed!80] ($(current page.north)+(-3,-8)$) rectangle ++(15,1.8);

  \end{scope}

  \begin{scope}[transform canvas ={rotate around ={45:($(current page.north west)+(4,-15.5)$)}}]

    \shade[rounded corners=25pt, left color=RedOrange, right color=DarkOrange] ($(current page.north west)+(4,-15.5)$) rectangle ++(30,1.8);

  \end{scope}

  \begin{scope}[transform canvas ={rotate around ={45:($(current page.north west)+(13,-10)$)}},]

    \shade[rounded corners=22pt, left color=DeepBlue,right color=DarkGreenBlue] ($(current page.north west)+(13,-10)$) rectangle ++(15,1.5);

  \end{scope}

  \begin{scope}[transform canvas ={rotate around ={45:($(current page.north west)+(18,-8)$)}},]

    \shade[rounded corners=8pt, left color=DarkOrange] ($(current page.north west)+(18,-8)$) rectangle ++(15,0.6);

  \end{scope}

  \begin{scope}[transform canvas ={rotate around ={45:($(current page.north west)+(19,-5.65)$)}},]

    \shade[rounded corners=12pt, left color=RedOrange] ($(current page.north west)+(19,-5.65)$) rectangle ++(15,0.8);

  \end{scope}

  \begin{scope}[transform canvas ={rotate around ={45:($(current page.north west)+(20,-9)$)}}]

    \shade[rounded corners=20pt, left color=BrightRed, right color=BrightRed!80] ($(current page.north west)+(20,-9)$) rectangle ++(14,1.2);

  \end{scope}

  \draw[ultra thick,gray] ($(current page.center)+(5,2)$) -- ++(0,-3cm) node[midway,left=0.25cm,text width=5cm,align=right,black!75]{{\fontsize{25}{30} \selectfont \bf RT\\[10pt] Bericht}} node[midway,right=0.25cm,text width=6cm,align=left,orange]{{\fontsize{70}{86} \selectfont 2021}};

  \node at ($(current page.center)+(0,-4)$) {{\fontsize{40}{72} \selectfont Regelungstechnik}};

  \node[text width=8cm,align=center] at ($(current page.center)+(0,-6.5)$) {{\fontsize{16}{20} \selectfont \textcolor{orange}{ \bf \today}} \\[3pt] Florian Tietjen 2519584\\[3pt] Emily Antosch 2519935};

\end{tikzpicture}

\newpage


\tableofcontents

\listoffigures

\newpage

\chapter{Die Temperaturregelung}


\section{Auswertung}


\subsection{Evaluation der Genauigkeit des stationären Verhaltens}

Um die Genauigkeit des stationären Verhaltens des Systems zu überprüfen, nutzen wir den Endwertsatz der Laplace-Transformation. Wir bezeichnen einen Regelkreis als stationär genau, wenn die Regeldifferenz des stationären Endwertes gegen Null läuft.
\noindent
Wir verwenden die allgemeine Formel:

\begin{equation}
  \label{eq:endwert}
  e(\infty) = \lim_{s\to 0} s \cdot \frac{1}{1+G_{o}(s)}\cdot X(s)
\end{equation}
\noindent
Die Übertragungsfunktion des offenen Regelkreises $G_{o}$ haben wir in der Vorbereitung verwendet:

\begin{equation}
  \label{eq:1}
  G_{o}(s) = K_P\frac{1+T_ns}{T_ns}\cdot K_{ges}\frac{1}{(1+T_Ss)(1+T_ts)}
\end{equation}
\noindent
Mit der Vereinfachung $T_{S} = T_{n}$ gilt dann:

\begin{equation}
  \label{eq:3}
  G_{o}(s) = \frac{K_{P}\cdot K_{ges}}{(1+T_{t}s)\cdot s \cdot T_{S}}
\end{equation}
Eingesetzt in die Formel (\ref{eq:endwert}) ergibt sich:

\begin{equation}
  \label{eq:2}
  e(\infty) = \lim_{s\to 0} s \cdot \frac{1}{1+\frac{K_{P}\cdot K_{ges}}{(1+T_{t}s)\cdot s \cdot T_{S}}} \cdot X(s)
\end{equation}
Für $X(s)$ setzen wir einen Sprung als Anregung für die Führungs- und Störgröße und erhalten:

\begin{equation}
  \label{eq:4}
  e(\infty) = \lim_{s\to 0} s \cdot \frac{1}{1+\frac{K_{P}\cdot K_{ges}}{(1+T_{t}s)\cdot s \cdot T_{S}}} \cdot\frac{1}{s} = \frac{T_{t}\cdot T_{S} \cdot s^{2} + T_{S}s}{T_{t}\cdot T_{S} \cdot s^{2} + T_{S}s+K_{P}\cdot K_{ges}} = 0
\end{equation}
Es ist ersichtlich, dass der Regler stationär genau ist, was für einen Integrator-Glied, welches hier enthalten ist, auch zu erwarten war. Ein Stör- oder Führungssprung führt trotzdem zu einem genauen stationären Endwert.

\newpage

\subsection{Überschwinger der Führungssprungantworten}

In diesem Schritt werden nun die Überschwinger der Führungssprungantworten ermittelt und dann miteinander verglichen

\begin{table}[h]
  \centering
  \begin{tabular}{lr}
    \hline
    Simulation/Messung & Überschwinger (in Prozent) \\
    \hline
    Simulation $PT_{2}$-Glied & $5\%$ \\
    Simulation $PT_{1}T_{t}$-Glied & $5,5\%$ \\
    \hline
  \end{tabular}
  \caption{Tabelle der ersten Überschwinger der Führungssprungantworten}
  \label{tab:lead}
\end{table}



\end{document}
