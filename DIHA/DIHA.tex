\documentclass{report}


\usepackage{circuitikz} %Für die Schaltpläne
\usepackage[T1]{fontenc}
\usepackage[utf8]{inputenc}
\usepackage{amsmath}
\usepackage{amssymb}
\usepackage{fancyhdr}
\usepackage{graphicx}
\usepackage{hyperref}
\usepackage{subcaption}
\usepackage{tikz}
\usepackage{cite}
\usepackage[nottoc, numbib]{tocbibind}
\usepackage{../assets/scripts/tex/color-env}
\usepackage[ngerman]{babel}


\usetikzlibrary{shapes}
    \usetikzlibrary{arrows}
    \usetikzlibrary{arrows.meta,topaths}
    \usetikzlibrary{bending}
    \usetikzlibrary{calc}
\title{Hausarbeit DI}


\usepackage[
  includehead,
  headheight = 17mm,
  footskip = \dimexpr\headsep+\ht\strutbox\relax,
  tmargin = 0mm,
  bmargin = \dimexpr17mm+2\ht\strutbox\relax,
]{geometry}

\usepackage{anyfontsize}

\usepackage{xcolor}

\definecolor{DarkGreenBlue}{HTML}{264653}
\definecolor{LightGreenBlue}{HTML}{2A9D8F}
\definecolor{LightOrange}{HTML}{E9C46A}
\definecolor{DarkOrange}{HTML}{F4A261}
\definecolor{RedOrange}{HTML}{E76F51}
\definecolor{BrightRed}{HTML}{D62828}
\definecolor{DeepBlue}{HTML}{003049}

\newcommand\invisiblesection[1]{%
  \refstepcounter{section}%
  \addcontentsline{toc}{section}{\protect\numberline{\thesection}#1}%
  \sectionmark{#1}}



\pagestyle{fancy}
\fancyhead[L]{\leftmark}
\fancyhead[R]{}
\fancyfoot[L]{}
\fancyfoot[C]{\thepage}
\fancyfoot[R]{\includegraphics[scale=0.2]{../assets/images/haw.jpg}}
\renewcommand\headrulewidth{0.5pt}


\begin{document}


\thispagestyle{empty}
\begin{tikzpicture}[remember picture,overlay]

  \fill[DeepBlue] (current page.south west) rectangle (current page.north east);

  \begin{scope}

    \foreach \i in {2.5,...,22}
      {
        \node[rounded corners, DeepBlue!90,draw ,regular polygon, regular polygon sides=6, minimum size=\i cm, ultra thick] at ($(current page.west)+(2.5,-5)$) {} ;
      }

  \end{scope}

  \node[rounded corners,fill=DeepBlue!95,text =DeepBlue!5,regular polygon,regular polygon sides=6, minimum size=2.5 cm,inner sep=0,ultra thick] at ($(current page.west)+(2.5,-5)$) {\LARGE \bfseries 2020};

  \foreach \i in {0.5,...,22}
    {
      \node[rounded corners,DeepBlue!90,draw,regular polygon,regular polygon sides=6, minimum size=\i cm,ultra thick] at ($(current page.north west)+(2.5,0)$) {} ;
    }

  \foreach \i in {0.5,...,22}
    {
      \node[rounded corners,DeepBlue!98,draw,regular polygon,regular polygon sides=6, minimum size=\i cm,ultra thick] at ($(current page.north east)+(0,-9.5)$) {} ;
    }

  \foreach \i in {12}
    {
      \node[fill = DeepBlue,rounded corners,draw=DeepBlue,regular polygon,regular polygon sides=6, minimum size=\i cm,ultra thick] at ($(current page.south east)+(-0.2,-0.45)$) {} ;
    }


  \foreach \i in {21,...,6}
    {
      \node[DeepBlue!95,rounded corners,draw,regular polygon,regular polygon sides=6, minimum size=\i cm,ultra thick] at ($(current page.south east)+(-0.2,-0.45)$) {} ;
    }

  \node[left,DeepBlue!5,minimum width=0.625*\paperwidth,minimum height=3cm, rounded corners] at ($(current page.north east)+(0,-9.5)$){{\fontsize{25}{30} \selectfont \bfseries DI - Hausarbeit}};

  \node[left,DeepBlue!10,minimum width=0.625*\paperwidth,minimum height=2cm, rounded corners] at ($(current page.north east)+(0,-11)$){{\huge \textit{Fehlersichere Übertragung und Speicherung}}};

  \node[left,DeepBlue!5,minimum width=0.625*\paperwidth,minimum height=2cm, rounded corners] at ($(current page.north east)+(0,-13)$){{\Large \textsc{Eric Antosch}}};

\end{tikzpicture}

\newpage


\tableofcontents

\listoffigures

\newpage

\chapter{Teil A - Plausibilitätsprüfung}
\label{cha:teil-plausibilitat}

\invisiblesection{Aufgabenstellung}
\label{sec:aufgabe}


\begin{task}
  TIn dem ersten Teilbereich der Hausarbeit soll das von einem externen Sensor erfasste 10 Bit breite Datenpacket auf hinreichende Abtastung mittels einer Plausibilitätsprüfung überprüft werden. Dabei soll das Signal $G_X$ überprüft und dann mit einem Signal $G_{XOK}$ dargestellt werden, dass das Signal zur Weiterverarbeitung übertragen werden kann.
\end{task}

\section{Blockschaltbild}
\label{sec:blockschaltbild}

Wir wollen nun zunächst das Blockschaltbild für unser Vorhaben erstellen, sodass wir bei Beschreibung des Systems in VHDL einen besseren Überblick über alle Signale und Komponenten haben.

\section{Lösungsidee}
\label{sec:losungsidee}

\section{Beschreibung in VHDL}
\label{sec:beschreibung-vhdl}

\section{Simulation der Ergebnisse}
\label{sec:simul-der-ergebn}

\section{Fazit}
\label{sec:fazit}


\chapter{Berechnung der Regelabweichung}
\label{cha:berechn-der-regel}




\end{document}
