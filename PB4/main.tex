\documentclass{report}


\usepackage[T1]{fontenc}
\usepackage[utf8]{inputenc}
\usepackage{amsmath}


\usepackage{enumerate}

\usepackage{graphicx}
\usepackage{fancyhdr}
\usepackage{lettrine}
\usepackage{hyperref}
\usepackage{subcaption}
\usepackage{tikz}
\usepackage{cite}
\usepackage{listings}
\usepackage[nottoc, numbib]{tocbibind}
\usepackage[ngerman]{babel}
\usepackage[Glenn]{fncychap}
\usepackage{trfsigns}
\usepackage{parskip}
\usepackage{microtype}


\usetikzlibrary{shapes}
    \usetikzlibrary{arrows}
    \usetikzlibrary{arrows.meta,topaths}
    \usetikzlibrary{bending}
    \usetikzlibrary{calc}
\title{Elektrotechnik 1 Praktikum 1}


\usepackage[
  includehead,
  headheight = 17mm,
  footskip = \dimexpr\headsep+\ht\strutbox\relax,
  tmargin = 0mm,
  bmargin = \dimexpr17mm+2\ht\strutbox\relax,
]{geometry}

\usepackage{anyfontsize}
\usepackage{float}
\usepackage{xcolor}

\definecolor{DarkGreenBlue}{HTML}{264653}
\definecolor{LightGreenBlue}{HTML}{2A9D8F}
\definecolor{LightOrange}{HTML}{E9C46A}
\definecolor{DarkOrange}{HTML}{F4A261}
\definecolor{RedOrange}{HTML}{E76F51}
\definecolor{BrightRed}{HTML}{D62828}
\definecolor{DeepBlue}{HTML}{003049}

\lstdefinestyle{code}{
    backgroundcolor=\color{backcolour},
    commentstyle=\color{codegreen},
    keywordstyle=\color{magenta},
    numberstyle=\tiny\color{codegray},
    stringstyle=\color{codepurple},
    basicstyle=\ttfamily\footnotesize,
    breakatwhitespace=false,
    breaklines=true,
    captionpos=b,
    keepspaces=true,
    numbers=left,
    numbersep=5pt,
    showspaces=false,
    showstringspaces=false,
    showtabs=false,
    tabsize=2
}

\definecolor{codegreen}{rgb}{0,0.6,0}
\definecolor{codegray}{rgb}{0.5,0.5,0.5}
\definecolor{codepurple}{rgb}{0.502,0.502,0.0}
\definecolor{backcolour}{rgb}{0.95,0.95,0.95}

\lstdefinelanguage{ST}
{
	morekeywords={
	case,of,if,then,end_if,end_case,super,function_block,extends,var,
	constant, byte,,end_var,var_input, real,bool,var_output,
	dint,udint,word,dword,array, of,uint,not,adr, program, for, end_for, while, do, end_while, repeat, end_repeat, until, to, by, else, elsif
	},
	otherkeywords={
		:, :=, <>,;,\,.,\[,\],\^,1,2,3,4,5,6,7,8,9,0, TRUE, FALSE, \{attribute,  \'hide\'\}
	},
	keywords=[1]{
		case,of,if,then,end_if,end_case,super,function_block,extends,var,
		constant, byte,,end_var,var_input, real,bool,var_output,
		dint,udint,word,dword,array, of,uint,not,adr, :, :=, <>,;,\,.,\[,\],\^,program, for, end_for, while, do, end_while, repeat, end_repeat, until, to, by, else, elsif
	},
	keywordstyle=[1]\color{blue},
	keywords=[2]{
		1,2,3,4,5,6,7,8,9,0, TRUE, FALSE
	},
	keywordstyle=[2]\color{codepurple},
	keywords=[3]{
		\{attribute,  \'hide\'\}
	},
	keywordstyle=[3]\color{codegray},
	sensitive=false,
	morecomment=[l]{//},
	morecomment=[s]{(*}{*)},
	morestring=[b]"
	morestring=[b]'
}

\lstset{
	language={ST},
	backgroundcolor=\color{backcolour},
	commentstyle=\color{codegreen}\textit,
	keywordstyle=\color{blue},
	numberstyle=\tiny\color{codegray},
	stringstyle=\color{codepurple},
	basicstyle=\ttfamily\scriptsize,
	breakatwhitespace=false,
	breaklines=true,
	captionpos=b,
	keepspaces=true,
	numbers=left,
	numbersep=5pt,
	showspaces=false,
	showstringspaces=false,
	showtabs=false,
	tabsize=2
}
\pagestyle{fancy}
\fancyhead[L]{\leftmark}
\fancyhead[R]{}
\fancyfoot[L]{}
\fancyfoot[C]{\thepage}
%\fancyfoot[R]{\includegraphics[scale=0.2]{../assets/images/haw.jpg}}
\renewcommand\headrulewidth{0.5pt}


\begin{document}


\thispagestyle{empty}
\begin{tikzpicture}[overlay,remember picture]
  \thispagestyle{empty}
  \fill[black!2] (current page.south west) rectangle (current page.north east);

  \begin{scope}[transform canvas ={rotate around ={45:($(current page.north west)+(-.5,-6)$)}}]

    \shade[rounded corners=18pt, left color=DarkGreenBlue, right color=LightGreenBlue] ($(current page.north west)+(-.5,-6)$) rectangle ++(9,1.5);

  \end{scope}

  \begin{scope}[transform canvas ={rotate around ={45:($(current page.north west)+(.5,-10)$)}}]

    \shade[rounded corners=18pt, left color=LightOrange,right color=DarkOrange] ($(current page.north west)+(0.5,-10)$) rectangle ++(15,1.5);

  \end{scope}

  \begin{scope}[transform canvas ={rotate around ={45:($(current page.north west)+(0.5,-10)$)}}]

    \shade[rounded corners=8pt, right color=DarkOrange, left color=LightOrange] ($(current page.north west)+(1.5,-9.55)$) rectangle ++(7,.6);

  \end{scope}

  \begin{scope}[transform canvas ={rotate around ={45:($(current page.north)+(-1.5,-3)$)}}]

    \shade[rounded corners=12pt, left color=DeepBlue!80, right color=DeepBlue!60] ($(current page.north)+(-1.5,-3)$) rectangle ++(9,0.8);

  \end{scope}

  \begin{scope}[transform canvas ={rotate around ={45:($(current page.north)+(-3,-8)$)}}]

    \shade[rounded corners=28pt, left color=BrightRed, right color=BrightRed!80] ($(current page.north)+(-3,-8)$) rectangle ++(15,1.8);

  \end{scope}

  \begin{scope}[transform canvas ={rotate around ={45:($(current page.north west)+(4,-15.5)$)}}]

    \shade[rounded corners=25pt, left color=RedOrange, right color=DarkOrange] ($(current page.north west)+(4,-15.5)$) rectangle ++(30,1.8);

  \end{scope}

  \begin{scope}[transform canvas ={rotate around ={45:($(current page.north west)+(13,-10)$)}},]

    \shade[rounded corners=22pt, left color=DeepBlue,right color=DarkGreenBlue] ($(current page.north west)+(13,-10)$) rectangle ++(15,1.5);

  \end{scope}

  \begin{scope}[transform canvas ={rotate around ={45:($(current page.north west)+(18,-8)$)}},]

    \shade[rounded corners=8pt, left color=DarkOrange] ($(current page.north west)+(18,-8)$) rectangle ++(15,0.6);

  \end{scope}

  \begin{scope}[transform canvas ={rotate around ={45:($(current page.north west)+(19,-5.65)$)}},]

    \shade[rounded corners=12pt, left color=RedOrange] ($(current page.north west)+(19,-5.65)$) rectangle ++(15,0.8);

  \end{scope}

  \begin{scope}[transform canvas ={rotate around ={45:($(current page.north west)+(20,-9)$)}}]

    \shade[rounded corners=20pt, left color=BrightRed, right color=BrightRed!80] ($(current page.north west)+(20,-9)$) rectangle ++(14,1.2);

  \end{scope}

  \draw[ultra thick,gray] ($(current page.center)+(5,2)$) -- ++(0,-3cm) node[midway,left=0.25cm,text width=5cm,align=right,black!75]{{\fontsize{25}{30} \selectfont \bf PBP4\\[10pt] Bericht}} node[midway,right=0.25cm,text width=6cm,align=left,orange]{{\fontsize{70}{86} \selectfont 2023}};

  \node at ($(current page.center)+(0,-4)$) {{\fontsize{40}{72} \selectfont Prozessbusleitsysteme}};

  \node[text width=8cm,align=center] at ($(current page.center)+(0,-6.5)$) {{\fontsize{16}{20} \selectfont \textcolor{orange}{ \bf \today}} \\[3pt] Florian Tietjen 2519584\\[3pt] PF: Emily Antosch 2519935 \\[3pt] Karl Döring 2519590 \\[3pt]};

\end{tikzpicture}

\newpage

\tableofcontents

\listoffigures

\newpage

\chapter{Bussysteme}

\section{Einführung}
\label{sec:einfuhrung}

In diesem Labor geht es um die Kommunikation zwischen zwei Teilnehmern in einem Bussystem. Dabei wird in der Programmiersprache AWL ein einfaches Beispiel-System aufgebaut, in dem ein Transportband mit dem Motor \it{M} und den beiden Lichtschranken \it{L1} und \it{L4}. Die Lichtschranken stellen die Enden des Transportbandes dar. Während des normalen Betriebs werden Werkstücke, die sich auf dem Band befinden immer wieder hin und her bewegt, indem die Lichtschranken die Richtung des Motors umkehren. Zudem gibt es einen softwareseitigen Notaus, der den Motor stoppt und das Band anhält.\\
Die Ziele für dieses Labor sind: 
\begin{itemize}
    \item Die Kommunikation zwischen zwei Teilnehmern in einem Bussystem verstehen
    \item Konfiguration eines Bussystems und dessen Betriebs
    \item Erstellen eines Programms in AWL
    \item Einbinden von zusätzlichen Teilnehmern in ein bestehendes Bussystem
    \item Aufbau und Inbetriebnahme eines PROFIBUS DP Netzwerks
    \item Aufbau und Inhalt von PROFIBUS DP Telegrammen verstehen
\end{itemize}

\section{Vorbereitung}
Zur Vorbereitung soll nun ein Grundverständnis für das bestehende Steuerungsprogramm geschaffen werden. Dazu wird das Programm in AWL analysiert und die einzelnen Schritte erklärt.

\subsection{OB100}
\label{sec:ob100}
Dieser Organisationsbaustein wird nur einmalig zur Initialisierung aufgerufen.

\begin{lstlisting}
SET
S M4.0 
L 0
T AW260
\end{lstlisting}
Hier wird der Notaus, der in der Variable \it{M4.0} gespeichert ist, gesetzt. Dadurch ist der Notaus nicht gedrückt, da er drahtbruchsicher implementiert ist. Darüber hinaus wird die Bandgeschwindigkeit, die in AW260 ist, auf 0 gesetzt.

\subsection{OB1}
Dieser Organisationsbaustein wird zyklisch aufgerufen und ist in Netzwerke unterteilt.

\subsubsection{Netzwerk 1}

\begin{lstlisting}
    UN M4.0 
    SPBN M001 
    L 0
    T AW260
    SPB M004 
M001: NOP 0
\end{lstlisting}
Dieses Netzwerk überprüft, ob der Notaus gedrückt ist. Wenn dies der Fall ist, wird der Motor ausgeschaltet. Wenn der Notaus nicht gedrückt ist, bleibt der Zustand des Motors unverändert.

\subsubsection{Netzwerk 2}

\begin{lstlisting}
UN E0.2 
R A0.1
SPBN M002 
L 10000 
T AW260
M002: NOP 0
\end{lstlisting}
Dieses Netzwerk überprüft die Lichtschranke \it{L1}. Wenn diese unterbrochen ist, wird die Drehrichtung des Motor umgekehrt, sodass das Werkstück nun wieder in Richtung der anderen Lichtschranke fährt. Wenn die Lichtschranke nicht unterbrochen ist, bleibt der Zustand des Motors unverändert. Wenn der Motor zu Anfang noch nicht gestartet war, wird die volle Drehgeschwindigkeit eingestellt und der Motor gestartet.

\subsubsection{Netzwerk 3}

\begin{lstlisting}
UN E0.3
S A0.1
SPBN M003
L 10000
T AW260
M003: NOP 0
\end{lstlisting}

Dieses Netzwerk überprüft die Lichtschranke \it{L4}. Wenn diese unterbrochen ist, wird die Drehrichtung des Motor umgekehrt, sodass das Werkstück nun wieder in Richtung der anderen Lichtschranke fährt. Wenn die Lichtschranke nicht unterbrochen ist, bleibt der Zustand des Motors unverändert. Wenn der Motor zu Anfang noch nicht gestartet war, wird die volle Drehgeschwindigkeit eingestellt und der Motor gestartet.
\subsubsection{Netzwerk 4}

\begin{lstlisting}
M004: NOP 0
\end{lstlisting}
Dieses Netzwerk stellt das Ende des Programms dar. Jeder Sprung zu diesem Netzwerk führt zum Ende des Programms.

\section{Aufbau}

Zunächst wird der PC gestartet und die Software \it{TIA-Portal} gestartet. Das vorgefertigte Projekt wird geöffnet und die Variablen sowie die OBs aus der Vorbereitung hinzugefügt. Nun wird die Konfiguration getestet und das Programm auf die CPU geladen.
Sollte das Programm nun vollständig funktionieren, werden nun die folgenden Schritte zur Analyse von Telegrammen auf einem PROFIBUS DP Netzwerk durchgeführt.


\section{Telegramme}



\end{document}
