\documentclass{article}


\usepackage{circuitikz} %Für die Schaltpläne
\usepackage[T1]{fontenc}
\usepackage[utf8]{inputenc}
\usepackage{subcaption}
\usepackage{amsmath}
\usepackage{fancyhdr}
\usepackage{lettrine}
\usepackage{hyperref}
\usepackage{subcaption}
\usepackage{tikz}
\usepackage{cite}
\usepackage{listings}
\usepackage[nottoc, numbib]{tocbibind}
\usepackage{../assets/scripts/tex/color-env}
\usepackage[ngerman]{babel}
%\input{../assets/scripts/tex/structure.tex}


\usetikzlibrary{shapes}
    \usetikzlibrary{arrows}
    \usetikzlibrary{arrows.meta,topaths}
    \usetikzlibrary{bending}
    \usetikzlibrary{calc}
\title{Elektrotechnik 1 Praktikum 1}


\usepackage[
  includehead,
  headheight = 17mm,
  footskip = \dimexpr\headsep+\ht\strutbox\relax,
  tmargin = 0mm,
  bmargin = \dimexpr17mm+2\ht\strutbox\relax,
]{geometry}

\usepackage{anyfontsize}

\usepackage{xcolor}

\definecolor{DarkGreenBlue}{HTML}{264653}
\definecolor{LightGreenBlue}{HTML}{2A9D8F}
\definecolor{LightOrange}{HTML}{E9C46A}
\definecolor{DarkOrange}{HTML}{F4A261}
\definecolor{RedOrange}{HTML}{E76F51}
\definecolor{BrightRed}{HTML}{D62828}
\definecolor{DeepBlue}{HTML}{003049}

\definecolor{codegreen}{rgb}{0,0.6,0}
\definecolor{codegray}{rgb}{0.5,0.5,0.5}
\definecolor{codepurple}{rgb}{0.58,0,0.82}
\definecolor{backcolour}{rgb}{0.95,0.95,0.92}

\lstdefinestyle{code}{
    backgroundcolor=\color{backcolour},
    commentstyle=\color{codegreen},
    keywordstyle=\color{magenta},
    numberstyle=\tiny\color{codegray},
    stringstyle=\color{codepurple},
    basicstyle=\ttfamily\footnotesize,
    breakatwhitespace=false,
    breaklines=true,
    captionpos=b,
    keepspaces=true,
    numbers=left,
    numbersep=5pt,
    showspaces=false,
    showstringspaces=false,
    showtabs=false,
    tabsize=2
}

\lstset{style=code}


\pagestyle{fancy}
\fancyhead[L]{\leftmark}
\fancyhead[R]{}
\fancyfoot[L]{}
\fancyfoot[C]{\thepage}
\fancyfoot[R]{\includegraphics[scale=0.2]{../assets/images/haw.jpg}}
\renewcommand\headrulewidth{0.5pt}


\begin{document}


\thispagestyle{empty}
\begin{tikzpicture}[overlay,remember picture]
  \thispagestyle{empty}
  \fill[black!2] (current page.south west) rectangle (current page.north east);

  \begin{scope}[transform canvas ={rotate around ={45:($(current page.north west)+(-.5,-6)$)}}]

    \shade[rounded corners=18pt, left color=DarkGreenBlue, right color=LightGreenBlue] ($(current page.north west)+(-.5,-6)$) rectangle ++(9,1.5);

  \end{scope}

  \begin{scope}[transform canvas ={rotate around ={45:($(current page.north west)+(.5,-10)$)}}]

    \shade[rounded corners=18pt, left color=LightOrange,right color=DarkOrange] ($(current page.north west)+(0.5,-10)$) rectangle ++(15,1.5);

  \end{scope}

  \begin{scope}[transform canvas ={rotate around ={45:($(current page.north west)+(0.5,-10)$)}}]

    \shade[rounded corners=8pt, right color=DarkOrange, left color=LightOrange] ($(current page.north west)+(1.5,-9.55)$) rectangle ++(7,.6);

  \end{scope}

  \begin{scope}[transform canvas ={rotate around ={45:($(current page.north)+(-1.5,-3)$)}}]

    \shade[rounded corners=12pt, left color=DeepBlue!80, right color=DeepBlue!60] ($(current page.north)+(-1.5,-3)$) rectangle ++(9,0.8);

  \end{scope}

  \begin{scope}[transform canvas ={rotate around ={45:($(current page.north)+(-3,-8)$)}}]

    \shade[rounded corners=28pt, left color=BrightRed, right color=BrightRed!80] ($(current page.north)+(-3,-8)$) rectangle ++(15,1.8);

  \end{scope}

  \begin{scope}[transform canvas ={rotate around ={45:($(current page.north west)+(4,-15.5)$)}}]

    \shade[rounded corners=25pt, left color=RedOrange, right color=DarkOrange] ($(current page.north west)+(4,-15.5)$) rectangle ++(30,1.8);

  \end{scope}

  \begin{scope}[transform canvas ={rotate around ={45:($(current page.north west)+(13,-10)$)}},]

    \shade[rounded corners=22pt, left color=DeepBlue,right color=DarkGreenBlue] ($(current page.north west)+(13,-10)$) rectangle ++(15,1.5);

  \end{scope}

  \begin{scope}[transform canvas ={rotate around ={45:($(current page.north west)+(18,-8)$)}},]

    \shade[rounded corners=8pt, left color=DarkOrange] ($(current page.north west)+(18,-8)$) rectangle ++(15,0.6);

  \end{scope}

  \begin{scope}[transform canvas ={rotate around ={45:($(current page.north west)+(19,-5.65)$)}},]

    \shade[rounded corners=12pt, left color=RedOrange] ($(current page.north west)+(19,-5.65)$) rectangle ++(15,0.8);

  \end{scope}

  \begin{scope}[transform canvas ={rotate around ={45:($(current page.north west)+(20,-9)$)}}]

    \shade[rounded corners=20pt, left color=BrightRed, right color=BrightRed!80] ($(current page.north west)+(20,-9)$) rectangle ++(14,1.2);

  \end{scope}

  \draw[ultra thick,gray] ($(current page.center)+(5,2)$) -- ++(0,-3cm) node[midway,left=0.25cm,text width=5cm,align=right,black!75]{{\fontsize{25}{30} \selectfont \bf GEP\\[10pt] Praktikum 2}} node[midway,right=0.25cm,text width=6cm,align=left,orange]{{\fontsize{70}{86} \selectfont 2021}};

  \node at ($(current page.center)+(0,-4)$) {{\fontsize{40}{72} \selectfont B6-Brücke}};

  \node[text width=8cm,align=center] at ($(current page.center)+(0,-6.5)$) {{\fontsize{16}{20} \selectfont \textcolor{orange}{ \bf 7. Dezember 2021}} \\[3pt] Emily Antosch 2519935};

\end{tikzpicture}

\newpage


\tableofcontents

\listoffigures

\listoftables


\newpage

\section{Einführung}

In diesem Versuch wollen wir uns mit der netzgeführten B6-Brücke beschäftigen. Dabei wollen wir sowohl eine ohmsche als auch eine ohmsch-induktive Last untersuchen und unsere Ergebnisse mit verschiedenen Messgeräten festhalten. 


\section{Vorbereitung}

Wir wollen uns zunächst über den Aufbau der B6-Brücke klar werden:
\begin{figure}[h]
  \centering
  \includegraphics[width=0.8\textwidth]{../assets/images/GEP2/b6.jpg}
  \caption{Aufbau der B6-Brücke}
  \label{fig:b6}
\end{figure}

Zusätzlich wollen wir uns im Vorfeld überlegen, inwieweit wir sicherstellen können, dass die vorgegebenen Werte eingehalten werden können. Mit $U_{S} = 26V$ und $I_{d,max} = 2A$ können wir nun bei maximaler Aussteuerung der Schaltung, also bei $\alpha = 0^{^{\circ}}$, die maximale Spannung $$U_{di\alpha} = \frac{3\cdot \sqrt{2}}{\pi}\cdot U_{L} \cdot cos(0^{\circ}) = \frac{3\cdot \sqrt{2}}{\pi}\cdot 26V = 60.816V$$
berechnen. Um nun eine ohmsche Last zu berechnen, die die Schaltung in diesen Werten beschränkt rechnen wir

\begin{equation*}
  \label{eq:1}
  R_{L} = \frac{U_{di\alpha}}{I_{d,max}} = \frac{60,816V}{2A} = 30.4 \Omega
\end{equation*}

\section{Messreihe}

\subsection{Messung der Steuerspannung und der gleichgerichteten Spannung}

Wir wollen zunächst unseren Offset bei der Einstellung unseres Zündverzögerungswinkels ermitteln. Dabei stellen wir unsere Steuerspannung $U_{St} = 10V$ auf das Maximum ein und messen vom Nulldurchgang der Spannung $U_{21}$ zur ersten Zündung. Wir erhalten eine Verzögerung von $\Delta t = 3.68ms$, damit rechnen wir

\begin{equation*}
  \label{eq:2}
  \Delta\alpha = \Delta t \cdot 360^{\circ} \cdot \frac{1}{T} - 60^{\circ} = 3.68ms \cdot 360^{\circ} \cdot \frac{1}{20ms} - 60^{\circ} = 6,2^{\circ}
\end{equation*}

und erhalten damit den Winkel, den wir bei der minimalen Einstellung unseres Zündwinkeltransformators haben. Alle weiteren Messungen basieren dann auf diesem Offset. Der Bild auf dem Oszilloskop ist dann unten noch einmal dargestellt:

\begin{figure}[h]
  \centering
  \includegraphics[width=\textwidth]{../assets/images/GEP2/startAngle.png}
  \caption{Startmessung des Winkels bei 10V}
  \label{fig:startAngle}
\end{figure}

\newpage

Wir wollen nun uns die Tabelle der Werte einmal anschauen:

\begin{table}[h]
  \centering
  \begin{tabular}{|c|c|c|}
    \hline
    $\alpha$ & $U_{St}$ & $U_{di\alpha}$ \\
    \hline
    $6,2^{\circ}$ & $56,6V$ & $10,008V$\\
    \hline
    $24,2^{\circ}$ &$53,01V$ & $9,058V$\\
    \hline
    $42,2^{\circ}$ &$44,1V$ & $8,059V$\\
    \hline
    $60,2^{\circ}$ & $31,6V$& $6,976V$\\
    \hline
    $78,2^{\circ}$ & $20,23V$& $6,131V$\\
    \hline
    $96,2^{\circ}$ & $8,96V$& $5,181V$\\
    \hline
    $114,2^{\circ}$ &$1,53V$ & $4,239V$\\
    \hline
    $132,2^{\circ}$ & $0,003V$& $3,3V$\\
    \hline
  \end{tabular}
  \caption{Messreihe der Steuerspannung und der gleichgerichteten Spannung im Bezug auf den Zündverzögerungswinkel}
  \label{tab:mess1}
\end{table}

Beispielhaft wollen wir uns dann auch das Oszilloskopbild mit den Spannungen $U_{21}$ und $U_{di\alpha}$ anschauen:

\begin{figure}[h]
  \centering
  \begin{subfigure}{.45\textwidth}
    \centering
    \includegraphics[width=\linewidth]{../assets/images/GEP2/31_Winkel242.png}
    \caption{Oszilloskopbild zur Messung 3.1 für den Winkel $24,2^{\circ}$}
  \end{subfigure}
  \begin{subfigure}{.45\textwidth}
    \centering
    \includegraphics[width=\linewidth]{../assets/images/GEP2/31_Winkel962.png}
    \caption{Oszilloskopbild zur Messung 3.1 für den Winkel $96,2^{\circ}$}
  \end{subfigure}
  \label{fig:31_242}
  \caption{Beispielhafte Bilder vom Oszilloskop}
\end{figure}


\subsection{Messung von verschiedenen Kenndaten der B6-Brücke bei ohmscher Last}

Im nächsten Schritt wollen wir uns unter der vorher berechneten ohmschen Last, also eine Zusammenschaltung von drei $100\Omega$-Widerständen, verschiedene Kenndaten der B6-Brücke anschauen. Auch hier schauen wir uns die Werte in Abhängigkeit von dem Zündverzögerungswinkel $\alpha$ ausgehend von unserem Offset in $18^{\circ}$-Schritten an. Dabei entsprechen $18^{\circ}$ genau einer Milisekunde Verzögerung.

\begin{table}[h]
  \centering
  \begin{tabular}{|c|c|c|c|c|c|}
    \hline
    $\alpha$ & $P_{zu}$ & $U_{S}$ & $I_{L}$ & $I_{d}$ & $U_{di\alpha}$\\
    \hline
    $6,2^{\circ}$ & $33,4W$ & $1,39A$ & $25,24V$ & $1,67A$ & $56,3V$\\
    \hline
    $24,2^{\circ}$ & $28,8W$ & $1,297A$ & $25,36V$ & $1,55A$ & $52,2V$\\
    \hline
    $42,2^{\circ}$ & $19,75W$ & $1,073A$ & $25,67V$ & $1,26A$ & $42,34V$\\
    \hline
    $60,2^{\circ}$ & $10,77W$ & $0,7083A$ & $25,9V$ & $0,866A$ & $28,55V$\\
    \hline
    $78,2^{\circ}$ & $4,1W$ & $0,49A$ & $26,3V$ & $0,4512A$ & $15,14V$ \\
    \hline
    $96,2^{\circ}$ & $0,728W$ & $0,217A$ & $26,39V$ & $0,1577A$ & $5,204V$\\
    \hline
    $114,2^{\circ}$ & $0W$ & $0A$ & $26,6V$ & $0,04A$ & $0,33V$\\
    \hline
    $132,2^{\circ}$ & $0W$ & $0A$ & $26,5V$ & $0A$ & $0,08V$\\
    \hline
  \end{tabular}
  \caption{Kenndaten der B6-Brücke bei ohmscher Last in tabellarischer Form}
  \label{tab:mess2}
\end{table}


Um unsere Messungen graphisch zu überprüfen, schauen wir uns die dazugehörigen Oszilloskopbilder auch an. Wir erkennen, dass ein Erhöhung des Zündverzögerungswinkels mit dem Absinken der Leistung einhergeht. Das entspricht auch unseren Vorstellungen. Auch die gleichgerichtete Spannung ist proportional zum Zündverzögerungswinkel.

\begin{figure}[h]
  \centering
  \begin{subfigure}{.45\textwidth}
    \centering
    \includegraphics[width=\linewidth]{../assets/images/GEP2/32_Winkel242.png}
    \caption{Oszilloskopbild zur Messung 3.2 für den Winkel $24,2^{\circ}$}
  \end{subfigure}
  \begin{subfigure}{.45\textwidth}
    \centering
    \includegraphics[width=\linewidth]{../assets/images/GEP2/32_Winkel602.png}
    \caption{Oszilloskopbild zur Messung 3.2 für den Winkel $60,2^{\circ}$}
  \end{subfigure}
  \label{fig:31_242}
  \caption{Zur Messung 3.2: Bilder vom Oszilloskop}
\end{figure}

\begin{figure}[h]
  \centering
  \begin{subfigure}{.45\textwidth}
    \centering
    \includegraphics[width=\linewidth]{../assets/images/GEP2/32_Winkel962.png}
    \caption{Oszilloskopbild zur Messung 3.2 für den Winkel $96,2^{\circ}$}
  \end{subfigure}
  \begin{subfigure}{.45\textwidth}
    \centering
    \includegraphics[width=\linewidth]{../assets/images/GEP2/32_Winkel1142.png}
    \caption{Oszilloskopbild zur Messung 3.2 für den Winkel $114,2^{\circ}$}
  \end{subfigure}
  \label{fig:31_242}
  \caption{Zur Messung 3.2: Bilder vom Oszilloskop}
\end{figure}

\subsection{Messung von verschiedenen Kenndaten der B6-Brücke bei ohmsch-induktiver Last}

Wir wiederholen jetzt die Messung mit einer ohmsch-induktiven Last, indem wir zu dem Widerstand noch eine Induktivität mit $L = 50mH$ in Reihe schalten.

\begin{table}[h]
  \centering
  \begin{tabular}{|c|c|c|c|c|c|}
    \hline
    $\alpha$ & $P_{zu}$ & $U_{S}$ & $I_{L}$ & $I_{d}$ & $U_{di\alpha}$\\
    \hline
    $6,2^{\circ}$ & $33,15W$ & $1,347A$ & $25,96V$ & $1,626A$ & $58,17V$ \\
    \hline
    $24,2^{\circ}$ & $27,99W$ & $1,236A$ & $26,03V$ & $1,493A$ & $53,26V$\\
    \hline
    $42,2^{\circ}$ & $17,75W$ & $0,977A$ & $26,15V$ & $1,192A$ & $42,26V$ \\
    \hline
    $60,2^{\circ}$ & $7,49W$ & $0,636A$ & $26,42V$ & $0,77A$ & $27,6V$ \\
    \hline
    $78,2^{\circ}$ & $0,936W$ & $0,213A$ & $26,4V$ & $0,26A$ & $9,13V$\\
    \hline
    $96,2^{\circ}$ & $55mW$ & $5mA$ & $26,58V$ & $2,2mA$ & $-0,378VV$\\
    \hline
    $114,2^{\circ}$ & $0W$ & $0A$ & $26,48V$ & $0,7mA$ & $0V$ \\
    \hline
    $132,2^{\circ}$ & $0W$ & $0A$ & $26,58V$ & $0,21mA$ & $0V$ \\
    \hline
  \end{tabular}
  \caption{Kenndaten der B6-Brücke bei ohmsch-induktiver Last in tabellarischer Form}
  \label{tab:mess2}
\end{table}

\begin{figure}[h]
  \centering
  \begin{subfigure}{.45\textwidth}
    \centering
    \includegraphics[width=\linewidth]{../assets/images/GEP2/32_Winkel242.png}
    \caption{Oszilloskopbild zur Messung 3.3 für den Winkel $24,2^{\circ}$}
  \end{subfigure}
  \begin{subfigure}{.45\textwidth}
    \centering
    \includegraphics[width=\linewidth]{../assets/images/GEP2/32_Winkel602.png}
    \caption{Oszilloskopbild zur Messung 3.3 für den Winkel $60,2^{\circ}$}
  \end{subfigure}
  \label{fig:31_242}
  \caption{Zur Messung 3.3: Bilder vom Oszilloskop}
\end{figure}

\begin{figure}[h]
  \centering
  \begin{subfigure}{.45\textwidth}
    \centering
    \includegraphics[width=\linewidth]{../assets/images/GEP2/32_Winkel962.png}
    \caption{Oszilloskopbild zur Messung 3.3 für den Winkel $96,2^{\circ}$}
  \end{subfigure}
  \begin{subfigure}{.45\textwidth}
    \centering
    \includegraphics[width=\linewidth]{../assets/images/GEP2/32_Winkel1142.png}
    \caption{Oszilloskopbild zur Messung 3.3 für den Winkel $114,2^{\circ}$}
  \end{subfigure}
  \label{fig:31_242}
  \caption{Zur Messung 3.3: Bilder vom Oszilloskop}
\end{figure}


\section{Auswertung}

\subsection{Kennlinie von Steuerspannung zu Zündwinkel}
\label{sec:kennl-von-steu}

Aus der Messreihe 3.1 wollen wir die Kennlinie $\alpha = f(U_{St})$ erstellen:


\subsection{Kennlinie von Winkel zu gleichgerichteten Spannung bei ohmscher Last}
\label{sec:kennlinie-von-winkel}


\subsection{Kennlinie von Winkel zu gleichgerichteter Spannung bei ohmsch-induktiver Last}
\label{sec:kennlinie-von-winkel-1}



\section{Konklusion}
\label{sec:konklusion}

Aus dem Labor konnten wir das Verhalten der B6-Brücke besser verstehen, indem wir verschiedene Schaltvorgänge und Verhalten der Schaltung genauer untersucht haben und selbst mit der Schaltung arbeiten konnten. Wir haben viele Zusammenhänge, die wir in der Vorlesung bereits behandelt haben, besser verinnerlichen können.


\end{document}
