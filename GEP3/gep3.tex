\documentclass{article}


\usepackage{circuitikz} %Für die Schaltpläne
\usepackage[T1]{fontenc}
\usepackage[utf8]{inputenc}
\usepackage{subcaption}
\usepackage{amsmath}
\usepackage{fancyhdr}
\usepackage{lettrine}
\usepackage{hyperref}
\usepackage{subcaption}
\usepackage{tikz}
\usepackage{cite}
\usepackage{listings}
\usepackage[nottoc, numbib]{tocbibind}
\usepackage{../assets/scripts/tex/color-env}
\usepackage[ngerman]{babel}
%\input{../assets/scripts/tex/structure.tex}


\usetikzlibrary{shapes}
    \usetikzlibrary{arrows}
    \usetikzlibrary{arrows.meta,topaths}
    \usetikzlibrary{bending}
    \usetikzlibrary{calc}
\title{Elektrotechnik 1 Praktikum 1}


\usepackage[
  includehead,
  headheight = 17mm,
  footskip = \dimexpr\headsep+\ht\strutbox\relax,
  tmargin = 0mm,
  bmargin = \dimexpr17mm+2\ht\strutbox\relax,
]{geometry}

\usepackage{anyfontsize}

\usepackage{xcolor}

\definecolor{DarkGreenBlue}{HTML}{264653}
\definecolor{LightGreenBlue}{HTML}{2A9D8F}
\definecolor{LightOrange}{HTML}{E9C46A}
\definecolor{DarkOrange}{HTML}{F4A261}
\definecolor{RedOrange}{HTML}{E76F51}
\definecolor{BrightRed}{HTML}{D62828}
\definecolor{DeepBlue}{HTML}{003049}

\definecolor{codegreen}{rgb}{0,0.6,0}
\definecolor{codegray}{rgb}{0.5,0.5,0.5}
\definecolor{codepurple}{rgb}{0.58,0,0.82}
\definecolor{backcolour}{rgb}{0.95,0.95,0.92}

\lstdefinestyle{code}{
    backgroundcolor=\color{backcolour},
    commentstyle=\color{codegreen},
    keywordstyle=\color{magenta},
    numberstyle=\tiny\color{codegray},
    stringstyle=\color{codepurple},
    basicstyle=\ttfamily\footnotesize,
    breakatwhitespace=false,
    breaklines=true,
    captionpos=b,
    keepspaces=true,
    numbers=left,
    numbersep=5pt,
    showspaces=false,
    showstringspaces=false,
    showtabs=false,
    tabsize=2
}

\lstset{style=code}


\pagestyle{fancy}
\fancyhead[L]{\leftmark}
\fancyhead[R]{}
\fancyfoot[L]{}
\fancyfoot[C]{\thepage}
\fancyfoot[R]{\includegraphics[scale=0.2]{../assets/images/haw.jpg}}
\renewcommand\headrulewidth{0.5pt}


\begin{document}


\thispagestyle{empty}
\begin{tikzpicture}[overlay,remember picture]
  \thispagestyle{empty}
  \fill[black!2] (current page.south west) rectangle (current page.north east);

  \begin{scope}[transform canvas ={rotate around ={45:($(current page.north west)+(-.5,-6)$)}}]

    \shade[rounded corners=18pt, left color=DarkGreenBlue, right color=LightGreenBlue] ($(current page.north west)+(-.5,-6)$) rectangle ++(9,1.5);

  \end{scope}

  \begin{scope}[transform canvas ={rotate around ={45:($(current page.north west)+(.5,-10)$)}}]

    \shade[rounded corners=18pt, left color=LightOrange,right color=DarkOrange] ($(current page.north west)+(0.5,-10)$) rectangle ++(15,1.5);

  \end{scope}

  \begin{scope}[transform canvas ={rotate around ={45:($(current page.north west)+(0.5,-10)$)}}]

    \shade[rounded corners=8pt, right color=DarkOrange, left color=LightOrange] ($(current page.north west)+(1.5,-9.55)$) rectangle ++(7,.6);

  \end{scope}

  \begin{scope}[transform canvas ={rotate around ={45:($(current page.north)+(-1.5,-3)$)}}]

    \shade[rounded corners=12pt, left color=DeepBlue!80, right color=DeepBlue!60] ($(current page.north)+(-1.5,-3)$) rectangle ++(9,0.8);

  \end{scope}

  \begin{scope}[transform canvas ={rotate around ={45:($(current page.north)+(-3,-8)$)}}]

    \shade[rounded corners=28pt, left color=BrightRed, right color=BrightRed!80] ($(current page.north)+(-3,-8)$) rectangle ++(15,1.8);

  \end{scope}

  \begin{scope}[transform canvas ={rotate around ={45:($(current page.north west)+(4,-15.5)$)}}]

    \shade[rounded corners=25pt, left color=RedOrange, right color=DarkOrange] ($(current page.north west)+(4,-15.5)$) rectangle ++(30,1.8);

  \end{scope}

  \begin{scope}[transform canvas ={rotate around ={45:($(current page.north west)+(13,-10)$)}},]

    \shade[rounded corners=22pt, left color=DeepBlue,right color=DarkGreenBlue] ($(current page.north west)+(13,-10)$) rectangle ++(15,1.5);

  \end{scope}

  \begin{scope}[transform canvas ={rotate around ={45:($(current page.north west)+(18,-8)$)}},]

    \shade[rounded corners=8pt, left color=DarkOrange] ($(current page.north west)+(18,-8)$) rectangle ++(15,0.6);

  \end{scope}

  \begin{scope}[transform canvas ={rotate around ={45:($(current page.north west)+(19,-5.65)$)}},]

    \shade[rounded corners=12pt, left color=RedOrange] ($(current page.north west)+(19,-5.65)$) rectangle ++(15,0.8);

  \end{scope}

  \begin{scope}[transform canvas ={rotate around ={45:($(current page.north west)+(20,-9)$)}}]

    \shade[rounded corners=20pt, left color=BrightRed, right color=BrightRed!80] ($(current page.north west)+(20,-9)$) rectangle ++(14,1.2);

  \end{scope}

  \draw[ultra thick,gray] ($(current page.center)+(5,2)$) -- ++(0,-3cm) node[midway,left=0.25cm,text width=5cm,align=right,black!75]{{\fontsize{25}{30} \selectfont \bf GEP\\[10pt] Praktikum 2}} node[midway,right=0.25cm,text width=6cm,align=left,orange]{{\fontsize{70}{86} \selectfont 2021}};

  \node at ($(current page.center)+(0,-4)$) {{\fontsize{40}{72} \selectfont B6-Brücke}};

  \node[text width=8cm,align=center] at ($(current page.center)+(0,-6.5)$) {{\fontsize{16}{20} \selectfont \textcolor{orange}{ \bf 4. Januar}} \\[3pt] Emily Antosch 2519935 \\[3pt] Florian Tietjen \\[3pt] Karl Döring};

\end{tikzpicture}

\newpage


\tableofcontents

\listoffigures

\listoftables


\newpage

\section{Einführung}

Dieser Laborbericht zum dritten Praktikum in Grundlagen der Energietechnik befasst sich  mit den Eigenschaften von Transformatoren im Kontext der Energietechnik. Dabei wird ein Einphasentransformator untersucht und die Parameter des Ersatzschaltbildes ermittelt.

Im Allgemeinen nutzt man Transformatoren zur Änderung des Spannungspegels von Wechselspannungen. Die Änderung der Spannung von der Primärseite zur Sekundärseite ist dabei direkt proportional zum Übersetzungsverhältnis. Dieses wird durch die Menge an Wicklungen um einen gemeinsamen Eisenkern bestimmt. Dieser zur Verbesserung der Induktion, was zu einem besseren Wirkungsfaktor führt. Über die Eigenschaft von elektrischen Strömen in Leitern Magnetfelder zu erzeugen wird von der Primärseite eine Spannung in der Sekundärseite induziert, welche dem vorher genannten Übersetzungsverhältnis entspricht.

Um bestimmte physikalische Prozesse, die zu Verlusten bei der Transformation der Spannung entstehen, besser im elektrotechnischen Kontext beschreiben zu können, wird ein allgemeines Ersatzschaltbild verwendet. Dabei beziehen sich die verschiedenen Größen auf die Primärseite. Alle Bauteile mit einer $1$ im Index ist auf der Primärseite, alle Bauteile mit einer $2$ sind hingegen auf der Sekundärseite.

\begin{figure}[h]
  \centering
  \includegraphics[width=\textwidth]{../assets/images/gep3/esb.png}
  \caption{Ersatzschaltbild eines Transformators}
  \label{fig:esbtrafo}
\end{figure}


Die Messungen an unserem Transformator werden maßgeblich von seinen Kenndaten beeinflusst, die wir vom Typenschild im Labor ablesen. Dabei haben wir auf der Primärseite $U_{N} = 380V$ und $I_{N} = 9,5A$ und auf der Sekundärseite $U'_{N} = 220V$ und $I'_{N} = 16A$. Mit diesen Angaben können wir nun bestimmte Messungen die Parameter des ESB Stück für Stück bestimmen.


\section{Messung der Wicklungswiderstände beider Seiten}

Wir messen zunächst die Wicklungswiderstände beider Seiten, indem wir ein Ohmmeter an die jeweiligen Klemmen des Transformators anschließen. Die Werte ergeben sich zu:

\begin{equation*}
  \label{eq:1}
  R_{1} = 0,55\Omega, R_{2} = 0,3\Omega
\end{equation*}

\section{Leerlaufversuch am Transformator}

Beim Leerlaufversuch am Transformator wollen wir verschiedene Messungen vornehmen, indem wir auf der Sekundärseite des Transformators keine Last zu schalten, sodass die Klemmen offen sind. Lediglich ein Voltmeter zur Spannungsmessung wird hinzugeschaltet.

\begin{figure}[h]
  \centering
  \includegraphics[width=\textwidth]{../assets/images/gep3/leerlauf_aufbau.png}
  \caption{Aufbau des Leerlaufversuchs}
  \label{fig:leerlaufaufbau}
\end{figure}

\subsection{Messung $I_{10}$ und $P_{10}$}
\label{sec:messung-i_10}
\begin{table}[h]
  \centering
  \begin{tabular}{|c|c|c|}
    \hline
    $22V$ & & \\
    \hline
  \end{tabular}
  \caption{Strom und Leistung in Abhängigkeit von der Spannung $U_{10}$}
  \label{tab:fifp}
\end{table}

\subsection{Bestimmung des Spannungsverhätnisses}
\label{sec:best-des-spann}

Im Nennpunkt für $U_{1N} = 220V$ messen wir beide Seiten des Transformators. Die Primärseite weißt dabei eine Spannung von $U_{2} = $ auf, wodurch man mit

\begin{equation*}
  \label{eq:2}
  ü = \frac{U_{2}}{U_{1N}} =
\end{equation*}

einen Spannungsübertragungsverhältnis berechnen. Die Differenz zur gedachten Übertragung liegt an den Herstellungsdifferenzen der Transformatoren. Gleichzeitig ist zu bemerken, dass der im Labor verwendete Transformator schon relativ alt ist, wodurch möglicherweise bereits Schäden an der Wicklung entstanden sein könnten.


\subsection{Einschaltmoment der Primärseite}
\label{sec:einsch-der-prim}

Mithilfe eines Schaltwinkelstellers betrachtet man nun den Einschaltvorgang auf der Primärseite des Transformators. Dabei können wir unseren Schaltwinkelsteller auf einen bestimmten Winkel, eine Dauer in Perioden und die positive oder negative Halbwelle einstellen.

\section{Konklusion}
\label{sec:konklusion}



\end{document}
